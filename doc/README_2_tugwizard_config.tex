
\subsection{Wizard Configuration}

TUG Wizard includes some files that allow developers to configure the
generation process. Some of these files can be modified from the TUG Wizard
user interface. Others can only be modified using an external text editor.
%
These files can be found in {\tt config/} folder, and are briefly described
in the following. 

%%
%%
{\tt config/generation\_templates}: set of templates used during the
generation of testsuites and test projects.
\begin{itemize}
%
\item {\tt testsuitebase\_template}: testsuite base template.
\item {\tt mp\_testsuite\_template}: template for testsuites in which a new
  panel is launched for each test.
\item {\tt mp\_test\_template}: template for tests in which a new panel is
  launched for each test.
\item {\tt op\_testsuite\_template}: template for testsuites in which only
  one panel is used to execute all tests (for leaf nodes in the test
  projects structure).
\item {\tt op\_testsuite\_internal\_template}: template for testsuites in
  which only one panel is used to execute all tests (for internal nodes in
  the test projects structure).
\item {\tt op\_test\_template}: template for tests in which only one panel
  is used to execute all tests.
\item {\tt testsuite\_project\_main}: template for main file launching a testsuite.           
\item {\tt testsuite\_project\_pro}: template for Qt project file compiling
  a testsuite.
%
\end{itemize}


%%
%%
{\tt config/includes}: set of files in which the includes related to
different options selected in TUG Wizard are defined. These files can be
modified also from TUG Wizard user interface.
\begin{itemize}
%
\item {\tt boost\_signals\_include}: defines the lines to include Boost
  signals into a test project.
\item {\tt libsig\_signals\_include}: defines the lines to include Libsig
  signals into a test project.
\item {\tt gcov\_include}: defines the lines to include Gcov into a test
  project.
\item {\tt gprof\_include}: defines the lines to include Gprof into a test
  project.
\item {\tt tuglib\_include}: defines the lines to include TUGLib library
  into a test project.
%
\end{itemize}
  


\newpage

%%% Local variables:
%%% mode: latex
%%% TeX-master: "README.tex"
%%% ispell-local-dictionary: "american"
%%% coding: utf-8
%%% fill-column: 75
%%% TeX-parse-self: t
%%% TeX-auto-save: t
%%% End:
